\documentclass[3p]{elsarticle}

\usepackage{graphicx,tikz,tikz-3dplot,tabu,amsmath,siunitx,amssymb}
\usepackage[colorlinks=true,linkcolor=black, citecolor=blue, urlcolor=blue]{hyperref}
\usepackage{lineno}
\modulolinenumbers[5]

\usepackage{stix}
\usepackage{trimclip}
\usepackage{xspace}


\graphicspath{{./figures/}}

\newcommand{\Astar}{\ensuremath{A^{*}}}
\newcommand{\Ustar}{\ensuremath{U^{*}}}
\newcommand{\fstar}{\ensuremath{f^{*}}}
\newcommand{\natf}{$f_n$}
\newcommand{\fn}{\ensuremath{f_n}}
\newcommand{\velrat}{\ensuremath{\alpha_r^*}}
\newcommand{\freqrat}{\ensuremath{f_r^*}}
\newcommand{\totph}{\ensuremath{\phi_{total}}}
\newcommand{\Cy}{\ensuremath{C_y^{RMS}}}
\newcommand{\fCy}{\ensuremath{f^*_{C_y}}}
\newcommand{\fCystar}{\ensuremath{f^*_{C_y}}}
\newcommand{\gtwo}{\ensuremath{\Gamma_2}}
\newcommand{\wz}{\ensuremath{\omega_z^*}}
\newcommand{\js}[1]{{\textcolor{blue}{{\bf{\it{ **John: #1 **}}}}}}

\journal{Journal of Fluids and Structures}

%%%%%%%%%%%%%%%%%%%%%%%
%% Elsevier bibliography styles
%%%%%%%%%%%%%%%%%%%%%%%
%% To change the style, put a % in front of the second line of the current style and
%% remove the % from the second line of the style you would like to use.
%%%%%%%%%%%%%%%%%%%%%%%

%% Numbered
\bibliographystyle{model1-num-names}

%% Numbered without titles
%\bibliographystyle{model1a-num-names}

%% Harvard
%\bibliographystyle{model2-names.bst}\biboptions{authoryear}

%% Vancouver numbered
%\usepackage{numcompress}\bibliographystyle{model3-num-names}

%% Vancouver name/year
%\usepackage{numcompress}\bibliographystyle{model4-names}\biboptions{authoryear}

%% APA style
%\bibliographystyle{model5-names}\biboptions{authoryear}

%% AMA style
%\usepackage{numcompress}\bibliographystyle{model6-num-names}

%% `Elsevier LaTeX' style
%\bibliographystyle{elsarticle-num}
%%%%%%%%%%%%%%%%%%%%%%%

\begin{document}

\begin{frontmatter}

\title{Vibration reduction of a sphere through shear-layer
	control}

%% Group authors per affiliation:
\author{Thomas McQueen}
\corref{cor1}
\ead{thomas.mcqueen@monash.edu}

\author{Jisheng Zhao}

\author{John Sheridan}

\author{Mark C. Thompson}

\cortext[cor1]{Corresponding author}

\address{Fluids Laboratory for Aeronautical and Industrial Research (FLAIR), Department of Mechanical and Aerospace Engineering, Monash University, Melbourne, VIC 3800, Australia}



\begin{abstract}
To date, it has been shown that the vibration response of an elastically mounted sphere undergoing vortex-induced vibration (VIV) can be controlled by imposing rotary oscillations at frequencies close to the vibration frequency. 
Here, we demonstrate that high-frequency rotary oscillations can be used to directly influence shear-layer vortex shedding and consequently reduce vibration. 
This approach contrasts with aiming to directly target the large-scale wake structures, using lower frequency perturbations. 
The oscillation frequencies imposed were between five and thirty-five times the natural frequency of the system and the amplitude of the rotational velocities were only 10\% of the free-stream velocity. 
The effects of the rotary oscillations were found to vary significantly across sphere vibration modes. 
In the mode III transition regime significant attenuation of the vibration response was observed for a narrow band of rotary oscillation frequencies. 
Time-resolved particle image velocimetry revealed that the shear-layer vortex structures locked to the forcing frequency, where suppression of the vibration response occurred. 
Optimal tuning of the oscillation frequency reduced the vibration amplitude in the mode III transition regime by 84\%, with a rotational velocity amplitude of only 10\% of freestream. 
These results show low-amplitude shear-layer forcing is a promising method of more efficiently suppressing VIV of three-dimensional geometries.
\end{abstract}

\begin{keyword}
Flow-induced vibration \sep flow control
\end{keyword}

\end{frontmatter}

%\linenumbers

\section{Introduction}%
\label{sec:Intro}%
Flow-induced vibration (FIV) arises frequently in a broad range of
engineering situations. 
Without adequate consideration of its effects, FIV can result in detrimental structural damage or complete structural failure. 
As a result, the phenomenon has been extensively studied. 
While there are multiple sources of excitation leading to FIV, perhaps the most extensively researched, and one often encountered in practice, is vortex-induced vibration (VIV). 
VIV occurs due to a synchronisation between an object's natural frequency and its associated vortex shedding frequency. 
Although to a lesser extent than cylinders, the VIV of spheres has also been studied. 
Work over the past two decades by \citet{Williamson1997}, \citet{Govardhan1997}, \citet{Jauvtis2001}, \citet{Govardhan2005}, \citet{vanHout2010}, \citet{Behara2011}, \citet{Eshbal2012}, \citet{vanHout2013}, \citet{Krakovich2013}, \citet{Lee2013}, \citet{Behara2016}, \citet{Rajamuni2018a}, \citet{Sareen2018c}, and \citet{Eshbal2019}, amongst others, on tethered and elastically mounted spheres, has shown that the complex three-dimensional sphere wake enables sustained body vibration across a broad parameter space of reduced velocity, mass-damping parameter, and Reynolds number.

Due to the aforementioned potential for severe structural damage,
extensive research has also been conducted to develop both passive and active methods to reduce, or
even eliminate, the occurrence of VIV. This has been primarily aimed at circular cylinders but also to a lesser extent for spheres. Furthermore, for
active control strategies, researchers have demonstrated
that control can not only be used to suppress vibration, but also to
amplify it. Therefore, active control methods could 
enhance the energy generation potential of devices such as
the Vortex Induced Vibration Aquatic Clean Energy device (VIVACE)
\citep{Bernitsas2008}. 

Recent work on controlling sphere FIV was conducted by \citet{Sareen2018}. That research examined the effect of constant rotation 
on the vibration response of a single degree of freedom (1DOF)
elastically mounted sphere. They could suppress the vibration
response across the mode I, mode II, and mode III transition
regimes. Due to the effect of the Magnus force, there was a mean offset
of the sphere position from the centre, which increased with both
rotation ratio and reduced velocity. \Citet{vanhout2012} implemented
acoustic control (using speakers mounted to the wind tunnel walls) on
a tethered sphere at frequencies higher than the shear layer
instability. They were able to suppress the response in the mode I and
mode II regimes and amplify it in the mode III regime.
\citet{Sareen2018b,Sareen2019} found that imposing rotary
oscillations of the sphere at frequencies around that of the natural
frequency of the system could reduce the magnitude of VIV. \citet{McQueen2020} implemented rotary oscillations by
employing a feedback control system that used the sphere displacement
as the controller input. The controller allowed the 
phase between sphere displacement and rotation to be adjusted.

The studies of \citet{Sareen2018b,Sareen2019} and \citet{McQueen2020}
implemented rotary oscillations at, or close to, the vibration
frequency of the sphere. The control was effective because they could
lock the large-scale vortex shedding seen in the wake
to the rotary oscillation frequency, or affect the phase between the
large-scale vortex shedding and vibration. This change to the
large-scale vortex shedding altered the fluid force acting on the
sphere, and in turn the vibration response. While effective, the
question remains whether it is more efficient to interact with
this large-scale instability as opposed to other, smaller-scale flow
structures found in the sphere's wake.

For a fixed sphere, \citet{Sakamoto1990} collated previous results and
conducted their own experiments on vortex shedding in the wake. They
described the existence of a large-scale instability of the wake
(termed the {\em low-mode instability}), which has a
Strouhal number of 0.2 over a broad Reynolds number range beginning
from approximately $Re=300$. Concurrently, they described a
high-frequency instability (termed the {\em high-mode instability})
associated with vortex rings, formed in the shear layer separating
from the sphere, beginning at a Reynolds number of 800 and observed up to
$6\times10^4$ by \citet{Kim1988}, the highest Reynolds number they
tested. Unlike the low-mode instability, the high-mode instability
increases with Reynolds number. \js{does the frequency or amplitude increase with Reynolds number? } 
  
In addition to quantifying the two instabilities, \citet{Sakamoto1990}
also classified the wake patterns as a function of Reynolds number. In the range of interest in this study ($3900 \lesssim Re \lesssim 2.3\times10^4$), the flow separates from the sphere just prior to \ang{90}, forming a vortex sheet that becomes unstable as the flow convects downstream. 
This results in the
eventual periodic shedding of vortex loops. The large-scale low-mode
instability of the wake becomes apparent further downstream, where a
waviness of the wake is observable. \citet{Jang2007} conducted flow
visualisations at $Re = 5300$ and $Re = 1.1\times10^4$ showing where
the unsteadiness in the separated vortex sheet begins. They
observed that the separated, laminar, vortex sheet remains
axisymmetrically stable to a distance of approximately $1.2-1.3$
diameters downstream of the rear of the sphere for $Re = 5300$, and
only $0.5$ diameters downstream for $Re = 1.1\times10^4$. Similarly, \citet{Bakic2006} conducted flow visualisations over a large
Reynolds number range of
$2.2\times10^4 \leqslant Re \leqslant 4\times10^5$. For
$Re = 2.2\times10^4$, their visualisations show distinct vortex
structures forming in the separated shear layer just downstream of the
rear of the sphere. For $Re = 5\times10^4$, they noted that in
addition to vortex roll-up, a vortex pairing process just downstream
of separation occurred. \citet{Rodriguez2011} conducted direct
numerical simulations at $Re = 3700$ and described the dynamics of the
shear layer in more detail. They outlined how initial random
disturbances are amplified and propagate downstream in the shear
layer. This leads to the formation of vortices that end up being both
drawn into the recirculating zone behind the sphere as well as feeding
the turbulent wake. \citet{Yun2006} examined the wake of a sphere at
$Re = 1 \times10^4$ and described how the vortices generated by the
high-mode instability convect downstream to compose the large-scale
waviness of vortical structures in the wake. They suggested a link
between the high- and low-mode instabilities, observing that multiple
cycles of vortex rings, composing about half of a large-scale wake
cycle, tilt in the same direction due to a difference in velocity
around the sphere which is closely associated with the wall-pressure
distribution. For an elastically mounted sphere,
or a tethered sphere at least, \citet{vanHout2013} found a broad
spectral peak in the shear layer centred around the frequency of the high-mode instability observed
in the shear layer of a fixed sphere at the same Reynolds number,
indicating that the effect of the elastic mounting on the high-mode
instability is minimal. 
Evidently, the shear-layer dynamics are a prominent
feature of the fixed sphere wake for the Reynolds number range of
interest here. Past research has shown that the structure of the near
wake is largely a result of the presence of the high-mode instability.
However, whether there is any connections between the low-
and high-mode instabilities remains an unanswered question.

To date, the studies implementing rotary oscillations
\citep[e.g.,][]{Sareen2018b,Sareen2019,McQueen2020} have all attempted
to directly influence the large-scale instability in the wake. Here,
we aim to determine if it is possible to suppress vibration by
altering the characteristics of the high-mode shear-layer instability
to in turn affect the large-scale, lift inducing, vortices using
low-amplitude rotary oscillations at much higher frequencies than
previously implemented. We seek to do this
using rotary oscillations at five to thirty-five times
the natural frequency of the system at an amplitude of, typically,
only 10\% of free-stream velocity. First, the
vibration response is characterised for select reduced velocities in
the mode II and mode III transition regimes. Time-resolved particle
image velocimetry (TR-PIV) is then used to examine the effect of the
oscillatory forcing on the wake of both a fixed and an elastically
mounted sphere for several control conditions.

Throughout this paper, the effects of imposed rotation are often
compared to the standard vibration response without imposed rotation.
Hereafter, the response of the sphere with no imposed rotation will be
referred to as the `natural' response.

A comparison to the work of
\citet{Achenbach1974,Kim1988,Sakamoto1990}, and \citet{Yun2006} provides 
some context to the imposed forcing frequencies implemented 
here. These studies examined examined the flow instabilities in the wake of a fixed
sphere. Figure~\ref{fig:otherstudies} shows the variation of the two
instabilities observed in the wake of a sphere over $300 \lesssim Re \lesssim 7 \times 10 ^4$ using data collated from past studies. Furthermore, the range of rotation frequencies implemented here in the mode III transition regime ($\Ustar=15$) where significant attenuation of the vibration occurs,  and the instabilities identified for a fixed sphere at equivalent conditions are shown. The imposed rotation frequencies are in the range between the
two wake instabilities, with the most effective forcing frequency
($\freqrat=21$) being approximately 37\% of the high-mode instability.
%
\begin{figure}
	\centering
	\includegraphics{otherstudies.png} %otherstudies.m
	\caption{Low- and high-mode instabilities as observed by
		\cite{Achenbach1974} ($\smallblacktriangleleft$), \cite{Kim1988}
		($\smallblacktriangleright$), \cite{Sakamoto1990}
		($\mdblklozenge$), and \cite{Yun2006} ($\smblksquare$). The
		vertical black line shows the range of rotary oscillation
		frequencies implemented for $\Ustar=15$ in this study. The
		orange circle markers indicate the low- and high-mode
		instabilities identified in the wake of the fixed sphere without
		rotation at the same Reynolds number as for $\Ustar=15$.}
	\label{fig:otherstudies}                                            
\end{figure}
%

%%%%%

%Although the optimal rotation frequency is only 37\% the high-mode
%instability, it is significantly higher ($>700$\%) than the low-mode
%instability. Furthermore, as discussed, for an elastically mounted
%sphere undergoing VIV, the low-mode instability remains locked to the
%vibration frequency in the mode II regime. The imposed forcing
%frequency is therefore further separated from the frequency of
%large-scale vortex shedding (low-mode instability) seen in the wake of
%a sphere undergoing VIV than indicated in
%figure~\ref{fig:otherstudies}. 

\section{Experimental methodology}
\subsection{Fluid-structure system modelling}%
The system studied here has an elastically mounted sphere that
is free to vibrate only in the y-direction, transverse to the free-stream
flow. Rotation (for control) of the sphere is imposed about the z-axis
perpendicular to the direction of free-stream flow and the
free-vibration axis. Figure~\ref{fig:Schematic} illustrates the
fluid-structure interaction set-up looking down along the rotation
axis.
%
\begin{figure}
	\centering
	\includegraphics{Schematic.eps}
	\caption{Schematic of the experimental set-up highlighting the
		key parameters for the transverse VIV of a rotatory
		oscillating sphere. The hydro-elastic system is simplified
		as a 1DOF system constrained to move in the cross-flow
		direction. The axis of rotation is perpendicular to both the
		free-stream flow direction ($x$-axis) and the vibration axis
		($y$-axis). Here, $U$ is the free-stream velocity, $F_y$ is
		the transverse force, $k$ is the spring constant, $D$ is the
		sphere diameter, $m$ is the oscillating mass, $c$ is the
		structural damping, and $\theta$ is the angular position of
		the sphere.}
	\label{fig:Schematic}
\end{figure}

As the sphere was constrained to vibrate along one axis, the governing
equation of motion of the system can be expressed as
%
\begin{equation}\label{rotation}
m\ddot{y}+c\dot{y}+ky=F_y {,}
\end{equation}
%
where $m$ is the total oscillating mass, $c$ is the structural damping
factor, $k$ is the spring constant, $y$ is the body displacement, and
$F_y$ is the transverse fluid force (the transverse lift). The
vibration response of the sphere can be characterised using the
non-dimensional parameters listed in table~\ref{tab:Parameters}.
%
\begin{table}
	\renewcommand{\arraystretch}{1.5}% adjust the row spacing
	\begin{center}
		\def~{\hphantom{0}}
		\begin{tabular}{lccc}
			Amplitude ratio   &   \Astar & $\sqrt{2}A_{RMS}/D$\\
			Damping ratio & $\zeta$ & $c/\sqrt{k(m+m_A)}$\\
			Frequency ratio & \fstar & $f/f_n$\\
			Mass ratio & $m^*$ & $m/m_d$\\
			Mass-damping parameter & $\xi$ & $(m^*+C_A)\zeta$\\
			Reduced velocity & \Ustar & $U/f_nD$\\
			Reynolds number & $Re$ & $\rho UD/\mu$\\
			Strouhal number & $St$ & $f_{vo}D/U$\\
			Transverse force coefficient & $C_y$ & $F_y/(\frac{1}{8}\rho U^2\pi D^2)$\\
			Transverse force frequency ratio & $f^*_{C_y}$ & $f_{C_y}/f_n$\\
		\end{tabular}
	\end{center}                
	\caption{Relevant non-dimensional parameters. Here, \Astar~is the root-mean-square value of the vibration amplitude in the $y$ direction, $D$ is the sphere diameter, $c$ is the structural damping, $k$ is the structural stiffness, $m$ is the oscillating mass, $m_A=C_Am_d$ is the added mass, $m_d$ is the displaced mass of the fluid, $C_A$ is the added mass coefficient ($0.5$ for a sphere), $f_n$ is the natural frequency of the system in quiescent water, $f_{vo}$ is the equivalent fixed-body vortex shedding frequency, $f$ is the body oscillating frequency, and $F_y$ is the transverse fluid force acting on the sphere.}
	\label{tab:Parameters}
\end{table}

Equation \ref{rotation} defines the rotation profile of the sphere and
introduces the two parameters used to vary the rotary oscillations,
namely: rotation amplitude ($\Omega_0$) (peak angular velocity), and
rotary oscillation frequency ($f_r$).

\begin{equation}
\label{rotation} \Omega={\Omega_0}\sin{(2\pi{f_r}t)}.\\
\end{equation}

These two parameters are normalised as per equations
\ref{rotationratio} and \ref{forcingfreqratio}, and are referred to as
the rotation ratio (\velrat) and forcing frequency ratio (\freqrat),
respectively.

\begin{equation}
\label{rotationratio} {\alpha_r}^*=\frac{D\Omega_0}{2U}.\\
\end{equation}

\begin{equation}
\label{forcingfreqratio} {f_r}^*= \frac{f_r}{f_{n}}.
\end{equation}

The rotation ratio was kept constant at \velrat~$=0.1$, aside from
\S~\ref{sec:rotamp}, where it was varied between $0\leqslant$ \velrat~$\leqslant1$
to examine the efficiency of the control strategy. Likewise, the
forcing frequency ratio was varied between $1\leqslant$ \freqrat~$\leqslant35$,
except in \S~\ref{sec:rotamp}, where it was varied between $0.125\leqslant$
\freqrat~$\leqslant35$. The upper limit of these parameters was due to
torque limitations of the servo-motor.
%
\subsection{Experimental set-up}%
The investigation was conducted in the recirculating free-surface
water channel of the Fluids Laboratory for Aeronautical and Industrial
Research at Monash University. The water channel has a working section
of 600\,mm in width, 800\,mm in depth, and 4000\,mm in length. The
free-stream turbulence level was less than 1\% over the flow-rate
range investigated. A schematic of the experimental set-up is shown in
figure~\ref{fig:setup}. A 70\,mm diameter sphere, CNC precision
machined from modelling board (Renshape 460), was mounted by a 3\,mm
rod to a servo motor (Maxon Motor, EC-max 4-pole 22, equipped with a
rotary encoder with a resolution of 5000 counts per revolution). The
servo-motor was mounted to a linear air-bearing system that
constrained the sphere to move with only 1DOF, transverse to the
oncoming flow. The top of the sphere was immersed one sphere diameter
beneath the free-surface, a compromise between minimising the
influence of the mounting rod and free surface, suggested by
\citet{Govardhan2005} and \citet{Sareen2018c}. The transverse sphere position was
measured using a digital linear encoder (RGH24, Renishaw, UK) with
a resolution of $1\,\mu$m. Both the transverse and angular positions of
the sphere were measured at a sampling rate of 500\,Hz for 300\,s for
each data-set, which consisted of at least 80 vibration cycles. The transverse displacement measurements were filtered
using a fourth-order low-pass Butterworth filter with a cutoff
frequency of 1\,Hz to remove high-frequency noise. For more details on
the air-bearing system see \citet{Nemes2012} and \citet{Zhao2014}. For
more details on the experimental set-up used here, see
\citet{McQueen2020}.

The mass ratio (the ratio of the total oscillating mass to the mass of
displaced fluid) of the system was $m^*= 10.1$. The structural damping
(with consideration of the added mass effect) and natural frequency of
the system in quiescent water were measured to be
$\zeta=4.22\times10^{-3}$ and $f_n=0.269$\,Hz. The
free-stream velocity was varied from $56\, \mathrm{mm\,s^{-1}}$ to
$336\, \mathrm{mm\,s^{-1}}$, corresponding to a Reynolds number range
of $3.9\times10^3 \leqslant Re \leqslant 2.3\times10^4$. The reduced
velocity range was $3 \leqslant U^* \leqslant 18$.

As a result of the highly accurate digital displacement measurements,
the sphere's velocity and acceleration can be derived accurately. 
In turn, the lift force and phase between lift and
displacement can be determined. Using the same air-bearing system as
used here, \citet{Zhao2018} and \citet{Sareen2018} have verified this
methodology for the circular cylinder and sphere, respectively, by
making comparison to independent measurements of lift force obtained
using a force balance.

To reveal the influence of the rotary oscillations on the wake
dynamics for both a fixed and an elastically mounted sphere, TR-PIV was
acquired in the equatorial plane ($x$-$y$). The flow was seeded using
hollow micro-spheres (model Sphericel 110P8; Potters Industries Inc.)
with normal diameter 13 $\mu$m and specific weight 1.1 g cm$^{-3}$. A
high-speed camera (Dimax S4, PCO AG, Germany) with resolution 2016
$\times$ 2016 pixel$^2$ was used in conjunction with a 5 W continuous
laser (MLL-N-532 mm, CNI, China) that produced a 3 mm thick laser
sheet to capture the images. A 105 mm lens (Nikkon, Japan) was used to
obtain a magnification factor of 16.89 pixel mm$^{-1}$. In-house
cross-correlation software, originally developed by
\citet{fouras2008}, was used to correlate interrogation windows of
size 32 $\times$ 32 pixel$^2$ with an overlap of 50\% to obtain the
velocity fields. This corresponded to a velocity vector field of 125
$\times$ 125 vector$^2$ with a vector spacing of 0.014 D. For each
experimental configuration, two sets of data, comprising 6297 images
each, were acquired at 500 Hz. Concurrently, the sphere transverse and
angular positions were recorded at 4\,kHz to enable the sphere to be
accurately located in the images.

%
\begin{figure}                                         
	\centering                                     
	\includegraphics{setup_openloop_v2.png}        
	\caption{Schematic of the experimental set-up.}
	\label{fig:setup}                              
\end{figure}                                           
%

\section{The Vibration response}%
\label{sec:response}%
\subsection{The natural response}%
Due to a synchronisation between the forces induced on the sphere,
primarily by large-scale vortex shedding in the wake
\citep{Govardhan2005}, and the natural frequency of the
fluid-structure system, an elastically mounted sphere will experience
vibration. The vibration can be characterised into a series of modes,
initially identified by \citet{Jauvtis2001}. With increasing
free-stream velocity, the vortex shedding frequency of the sphere
approaches the natural frequency of the system. Synchronisation or
`lock-in' then occurs, resulting in a relatively sudden onset of
vibration (mode I). Subsequently, the frequency of vortex shedding
remains locked-in to the natural frequency of the system and vibration
continues over a broad reduced velocity range. Mode II is defined by a
transition in the phase difference between the sphere transverse displacement and fluid force (total phase) from near \ang{0}, through \ang{90}, to near
\ang{180}. 

Figure~\ref{fig:natres} shows the mean vibration amplitude
over the reduced velocity range investigated in this study. The
vertical bars show the mean of the top and bottom 10\% of vibrations.
A higher velocity mode of vibration (mode III) occurs above the
reduced velocity range investigated. Here, the vibration regime past
mode II is termed the `mode III transition regime'. Unlike for the circular cylinder, there is no distinct change in the wake patterns observed in the wake of the sphere with varying reduced velocity. Rather, as mentioned, there is only a slow transition in the total phase. As such, it is more difficult to define the range of the modes associated with sphere vibration. The modes of interest in this study are annotated in figure~\ref{fig:natres} along with blue shading indicating the approximate range of the transition regions.
%
\begin{figure}
	\centering
	\includegraphics{natresponse.png} %Best Reduction
	\caption{Natural vibration response of the sphere (black
		diamonds). The vertical bars show the mean of the top and
		bottom 10\% of vibrations. The orange circles show the
		maximum variation in the vibration amplitude observed for
		the range of forcing frequency ratios implemented. The
		forcing frequency ratios shown are $\freqrat=7$ at
		$\Ustar=6$, $\freqrat=34$ at $\Ustar=9$, $\freqrat=34$
		at $\Ustar=10$, $\freqrat=6$ at $\Ustar=12$,
		$\freqrat=21$ at $\Ustar=15$, and $\freqrat=22$ at
		$\Ustar=18$. The blue shading indicates the approximate location of the transition regions between modes of vibration.}
	\label{fig:natres}
\end{figure}
%
\subsection{Effect of control on the vibration response}%
To characterise the effect of high-frequency low-amplitude forcing on
the vibration response, rotary oscillations were imposed over the
range $5\leqslant$ $\freqrat\leqslant35$, in $\Delta\freqrat=1$
increments, at reduced velocities in the mode I, mode II, and
mode III transition regimes with $\velrat=0.1$. The orange
markers in figure~\ref{fig:natres} depict the maximum change in the
vibration amplitude across the range of forcing frequency ratios
investigated. As evident from figure~\ref{fig:natres}, with imposed
high-frequency oscillations, it is more difficult to alter the
vibration response in the mode I and mode II transition regimes,
where the natural vortex shedding frequency remains
close to the natural frequency of the system. A similar observation
was made by \citet{Sareen2018b} for imposed rotary
oscillations at much lower frequencies around the natural frequency of
the system. Past the peak of the mode II regime the vibration
suppression is rapidly enhanced until approximately $\Ustar=15$. To summarise the effect of the imposed rotary oscillations on the
vibration response of the sphere, a contour plot showing the
percentage change in vibration amplitude from the natural response
over the $\freqrat - \Ustar$ parameter space investigated is shown in
figure~\ref{fig:combo}. Line contours in increments of 50\% vibration amplitude alteration have been arbitrarily chosen to highlight significant amplitude variation. The orange markers indicate the conditions at
which TR-PIV, to be examined in \S~\ref{sec:wake}, was acquired.
This figure highlights the relatively narrow band of forcing
frequencies for which significant vibration suppression occurs in the
mode III transition regime. The range of forcing frequencies for which significant suppression occurs widens with increasing reduced velocity. It also appears that at low reduced
velocities, high-frequency oscillations slightly increase the
amplitude response. In this study we are particularly interested in determining the conditions for which the imposed rotation significantly suppresses the vibration and any associated effects on the wake structures. Therefore, hereafter we focus on results obtained past the peak of the mode II response, where significant vibration suppression was identified (figures~\ref{fig:natres} and~\ref{fig:combo}). 
%
\begin{figure}
	\centering
	\includegraphics{VIVchange.png} %ContourOfReductionPhasev2.m
	\caption{Contour plot of the variation in vibration amplitude
		from the natural response across the $\Ustar - \freqrat$ parameter space. The colour contours show the
		percentage change in vibration amplitude from the natural
		response. The orange markers indicate conditions at which
		TR-PIV was acquired.}
	\label{fig:combo}
\end{figure}
%

In the mode III transition regime, at $\Ustar=15$, there is significant attenuation of the amplitude response (up to 81\%) for a relatively narrow band of forcing frequency ratios centred around $\freqrat\approx22$ (figure~\ref{fig:U15}($a$)). 
The vertical bars in figure~\ref{fig:U15}($a$), which show the mean of the top and bottom 10\% of vibrations, can provide an indication of the periodicity of the amplitude response. 
Away from the suppressed regime, little variation in periodicity is seen. 
Within the suppressed regime though, much larger variation in periodicity and a slightly broader range in the spectral power of the vibration frequency can be observed (figure~\ref{fig:U15}($b$)), however in general, $\fstar\approx1.05$ remains dominant throughout. 
There is also variation in the frequency of the transverse force acting on the sphere. 
For $\freqrat=21$, where the maximum reduction in vibration amplitude is observed, there is a distinct lack of spectral power seen in the power spectral density (PSD) estimate of transverse force frequency at $\fCy=1$ (figure~\ref{fig:U15}($c$)). 
The time series of sphere displacement (figure~\ref{fig:U15}($e-h$)) shows that even where vibration is significantly suppressed, and large variation in vibration amplitude is seen over the test duration, the vibration frequency remains close to $\fstar=1.05$. 
For $\freqrat=21$, at times the vibration response was nearly completely suppressed before abruptly increasing over only a few vibration cycles (figure~\ref{fig:U15}($g$)). This behaviour was repeatedly observed in the suppressed regime. 

Where the vibrations are suppressed, there is significant variation in
the instantaneous total phase measured. Therefore, the
arithmetic mean may not provide a reasonable indication of the mean
total phase. For example, the arithmetic mean of $0^\circ$ and
$180^\circ$ is $90^\circ$ which is not a useful measure for our
purpose. A more suitable estimation may be obtained by calculating the
circular mean.

For a circular quantity such as total phase, the mean resultant vector
of the total phase distribution can be expressed as
%
\begin{equation}
\bar{\rho} = \frac{1}{n}\sum_{j=1}^{n}e^{i{\phi_{total}}_j},
\end{equation}
%
where $n$ is the total number of samples in a data-set. The resultant
vector can be used to obtain a mean phase angle,
%
\begin{equation}
\phi_{total} = Arg{(\bar{\rho})},
\end{equation}
%
and an indication of the variance of the angles,
%
\begin{equation}
Var(\phi_{total}) = 1 - |{\bar{\rho}|},
\end{equation}
%
where the minimum possible variance, 0, indicates that all angles are
equal and the maximum, 1, indicates that the angles are spread over
\ang{0} to \ang{360} and that there is no useful indication of a
mean phase angle.

For forcing frequencies outside of the suppressed regime there is no
appreciable variation in total phase and only slight variation in the
variance of the total phase. In the suppressed regime, while there
appears to be a significant change to the mean total phase, the
variance is very close to one indicating that the total phase is
continuously varying and that there is no practical representation of
a mean total phase.
%
\begin{figure}
	\centering
	\includegraphics{U15plot.png}
	\caption{Response of the sphere with imposed rotation at
		\Ustar~$=15$ as a function of forcing frequency ratio. ($a$) Variation of vibration amplitude, the vertical bars show the mean of the top and bottom 10\% of vibrations. The black dashed line shows the natural response. ($b,c$) PSD contour plots of the vibration frequency (\fstar) and
		transverse force frequency (\fCystar), the spectral power is normalised by the maximum value at each \freqrat~and is
		presented on a log 10 scale. ($d$) Variation in total phase (black diamonds) and circular variance of total phase
		(orange circles). The black and orange dashed lines show the total phase and circular variance respectively
		for the natural response. Time series of sphere displacement for ($e$) \freqrat~$=0$,
		($f$) \freqrat~$=16$, ($g$) \freqrat~$=21$, and ($h$)
		\freqrat~$=26$.}
	\label{fig:U15}
\end{figure}
%

At the highest reduced velocity investigated, $\Ustar=18$, a step
change in the amplitude response at $\freqrat=20$ was observed,
resulting in a vibration amplitude reduction of up to 84\%
(figure~\ref{fig:U18}($a$)). Whilst a broader power spectrum of
vibration frequency is observed for forcing frequencies above the step
change in amplitude response, once more, the dominant vibration
frequency remains fixed for all forcing frequencies at
$\fstar\approx1.05$ (figure~\ref{fig:U18}($b$)). Beyond the step
change, the peak in spectral power seen in the transverse force at
$\fCy\approx1$ is no longer visible, with most of the power situated
at lower frequencies. Note that even for the natural response, the
vibrations are noticeably less periodic at this high reduced velocity.
Figure~\ref{fig:U18}($f,g$) highlights the rapid change in amplitude
response seen between $\freqrat=18$ and $\freqrat=21$.
At $\freqrat=18$, there is only slight variation in the vibration
amplitude from the natural response. At $\freqrat=21$, however, the
vibration is significantly suppressed and a low frequency `pulsing' in
the vibration response can be observed (figure~\ref{fig:U18}($g$)).
For very high forcing frequencies, there is large scatter in the
amplitude response. Examining the time series of sphere displacement
at $\freqrat=30$ (figure~\ref{fig:U18}($h$)), a characteristic
response for very high forcing frequencies, reveals periods of almost
complete vibration suppression followed by relatively large vibration.
This transition from nearly completely to only minimally suppressed
vibration appears to happen sporadically. As for $\Ustar=15$, the
total phase correlates well with changes in the amplitude response.
Where there is minimal vibration suppression, there is no appreciable
change to the total phase. Between $\freqrat=5$ and $\freqrat=20$,
where the step change in the amplitude response occurs, the variance
of the total phase steadily increases. Beyond the step change the
variance remains close to one, indicating once more that in the
suppressed regime there is no practical mean total phase.
%
\begin{figure}
	\centering
	\includegraphics{U18plot.png}
	\caption{Response of the sphere with imposed rotation at
		$\Ustar=18$ as a function of forcing frequency ratio. Time
		series of sphere displacement for ($e$) $\freqrat=0$,
		($f$) $\freqrat=18$, ($g$) $\freqrat=21$, and ($h$)
		$\freqrat=30$. See figure~\ref{fig:U15} for further
		details.}
	\label{fig:U18}
\end{figure}
%

\subsection{Effect of rotary oscillation amplitude} \label{sec:rotamp}
%
The intent of this study is predominantly to examine the effect of
high-frequency rotary oscillations on the vibration response for a rotation amplitude much smaller than the free-stream speed ($\velrat=0.1$). However, examining the effect of varying the rotation amplitude is of additional interest
as it allows us to gain an understanding of both the potential
efficiency of the control method and an indication of the sensitivity
of the method to small variations in control conditions. Therefore, the effect of varying rotation amplitude in the mode III transition regime ($\Ustar=15$)
over the $\velrat - \freqrat$ parameter space of $0\leq\velrat\leq0.75$ and $5\leq\freqrat\leq35$ was examined (figure~\ref{fig:ampvar}).

For rotation amplitudes below $\velrat=0.075$, there is minimal
vibration suppression at any forcing frequency. By $\velrat=0.1$, the
vibrations are significantly suppressed over the regime
$20\lesssim\freqrat\lesssim25$, as analysed in detail. By $\velrat=0.3$, the vibrations are significantly
suppressed for all forcing frequencies. Thereafter, the amplitude
response remains relatively insensitive to an increase in velocity
ratio. This suggests that the effect of the imposed rotary
oscillations on the vibration response becomes saturated and that any
further increase in energy input, at least over the range of rotation
amplitudes investigated, will only serve to reduce control efficiency.
%
\begin{figure}
	\centering
	\includegraphics{avarU15.png} %
	\caption{Variation of vibration amplitude across the $\velrat -
		\freqrat$ parameter space for $\Ustar=15$. Colour-bar shows
		\Astar.}
	\label{fig:ampvar}
\end{figure}
%

\subsection{Variation of instantaneous vortex phase}
\citet{Sareen2018b,Sareen2019} found that by imposing rotary
oscillations at frequencies close to the natural frequency of the
system, the large-scale streamwise vortex structures that have been
shown to be the primary contributor to sustained VIV
\citep{Govardhan2005} lock to the imposed oscillation frequency,
uncoupling the fluid forcing from the natural frequency of the system
and suppressing vibration. Evidently, the phase difference between the sphere
displacement and the force on the sphere due to vortex dynamics
(vortex phase, $\phi_{vortex}$)  is an important indicator of the
vibration response. The significantly higher forcing
frequencies imposed in this study were not expected to cause a lock-on 
to the large-scale structures if the vibration was supressed. From figure~\ref{fig:U15}, shows that where the vibration was suppressed, the variance of total phase
significantly increased, suggesting that the transverse fluid force
acting on the sphere became less `locked' to the sphere vibration.
The peaks in the fluid force appear to occur more sporadically over a
vibration cycle. To examine the variance in more detail, histograms of
the vortex phase at $\Ustar=15$ and $\velrat=0.1$ are shown in
figure~\ref{fig:phasehist}. For the natural response, the vortex phase
is highly consistent and centred close to \ang{180}. Similarly, with
imposed rotation at $\freqrat=16$, and to a lesser extent
$\freqrat=26$, the vortex phase is centred close to \ang{180} and
varies only slightly from the mean phase angle. For $\freqrat=21$
however, the histogram shows a wide spread in the vortex phase. The
phase angle is predominantly in the range
$\ang{90} < \phi_{vortex} < \ang{270}$ with two broad peaks centred
around \ang{150} and \ang{225}. This suggests that the
imposed rotation causes the large lift-inducing vortex structures to
form either earlier or later in the vibration cycle, but not regularly
at the position in the cycle seen for the natural response.
In the suppressed regime, while some variation in the
frequency of transverse force acting on the sphere was observed, the
dominant vibration frequency of the sphere remained constant at the
value observed for the natural response. The variation of vortex phase
and consistent vibration frequency suggests
that the imposed rotation does not induce any lock-on behaviour of the
large-scale wake structures, as observed in previous studies. Yet
significant differences in the vibration response were observed,
suggesting that the high-frequency oscillations interact with the
sphere wake in a different manner to that seen previously. \js{Which previous study are you referring to here?} To reveal
how the imposed rotary oscillations interact with the wake to suppress
vibration, the wake structure was examined.

\begin{figure}
	\centering
	\includegraphics{phasehist.png} %ContourOfPower
	\caption{Histograms of vortex phase. The polar angle shows vortex phase. The radius shows the probability density. The bin width is \ang{2}. The orange line indicates the circular mean vortex phase angle.}
	\label{fig:phasehist}
\end{figure}
%

\section{Influence of rotary oscillations on the wake structure} \label{sec:wake}%
To elucidate why the high-frequency rotary oscillations suppress VIV in the mode III transition regime for a
narrow band of frequencies, TR-PIV data was acquired in the
equatorial ($x$-$y$) plane of both a fixed and an elastically mounted
sphere. It was acquired for the imposed control conditions indicated by the orange markers in figure~\ref{fig:combo}. While the sphere wake is
inherently three-dimensional, the equatorial plane provides a cut
through the vortex structures emanating from the sphere in the location at
which the effect of rotation is likely highest, due to the maximum
circumferential velocity of the sphere at the equator. In addition to the natural response in the mode III transition regime, particular focus was given to the three imposed forcing frequencies indicated by the markers with black outline in figure~\ref{fig:combo}. As shown in figure~\ref{fig:combo}, these three imposed forcing frequencies should provide a good indication of any changes to the wake structures observed in, and on either side of, the suppressed regime. Due to the difficulty associated with dynamic analysis in the wake of a moving object, the
wake of a fixed sphere is first examined, as it is hypothesised that
any significant variations to the near wake would likely be observed
regardless of the mounting condition.

\subsection{Fixed Sphere} \label{sec:fixed} 
Figure~\ref{fig:meanUvel} shows mean streamlines of the time-averaged wake
and principle Reynolds stresses, $R_{y'y'}$. Some asymmetry of the
wake is noticeable due to the small number ($\sim10$) of low-mode
shedding cycles in each TR-PIV data set. Separation occurs at
approximately \ang{90}, in good agreement with past studies: \citet{Yun2006},
\ang{90} at $Re=1\times10^4$; \citet{Tomboulides1993}, \ang{88} at
$Re=2\times10^4$; \citet{Constantinescu2003}, \ang{84} at
$Re=1\times10^4$; and \citet{Grandemange2014}, \ang{90} at
$Re=1.9\times10^4$. The time-averaged separation angle is not noticeably affected by
the imposed rotation. The wake closure distance is shortened for all
imposed rotation frequencies, by up to approximately 15\% for
$\freqrat=21$ and $\freqrat=26$. Although the wake closure distance
is the same for $\freqrat=21$ and $\freqrat=26$, the vortex cores
are located closer to $y/D=0$ (i.e., further inboard) for
\freqrat~$=21$, indicating a narrower wake. $R_{y'y'}$ increases with
imposed forcing close to the sphere in the shear-layer region, where
the vortex structures are shed and convect downstream for the natural response.
For all forcing frequencies, high $R_{y'y'}$ is also seen in the
centre of the wake for $x/D>1$, where the shear-layer structures begin
to noticeably break down and the low-mode instability becomes dominant
for a stationary sphere at this Reynolds number.
%
\begin{figure}
	\centering
	\includegraphics{meanURE.png} %
	\caption{Top: Mean streamlines and streamwise velocity
		determined from two sets of TR-PIV data for $\Ustar=15$ and ($a$)
		$\freqrat=0$, ($b$) $\freqrat=16$, ($c$) $\freqrat=21$,
		and ($d$) $\freqrat=26$. The orange cross indicates the
		wake closure ($WC$) point.
		Bottom: $R_{y'y'}$ for ($e$) $\freqrat=0$, ($f$)
		$\freqrat=16$, ($g$) $\freqrat=21$, and ($h$)
		$\freqrat=26$.}
	\label{fig:meanUvel}
\end{figure}
%

To quantify the narrowing of the wake, the width of the reverse flow region (i.e., $u<0$ ms$^{-1}$) was calculated for all streamwise velocity vector locations between the rear of the sphere and wake closure (figure~\ref{fig:width}). At all downstream locations prior to wake closure the reverse flow region was narrowest for $\freqrat=21$, the most effective forcing frequency in suppressing vibration of the control conditions analysed. 
%
\begin{figure}
	\centering
	\includegraphics{WakeWidth.png}
	\caption{The width of the wake ($Y_{ww}$), defined as the time-mean streamwise reverse flow region.}
	\label{fig:width}
\end{figure} 
 

For the natural response, close to the sphere in the shear layer, a
distinct spectral peak in the cross-stream ($y$-direction) velocity at the
high-mode instability ($St=3.87$) is observed. However, only slightly further
downstream, the sub-harmonic of the high-mode instability ($St=1.93$)
becomes dominant. Figure~\ref{fig:p1spatial}($a$) shows the spatial distribution of spectral power for these two dominant frequencies seen in the shear layer for the natural response. Figure~\ref{fig:p1spatial}($b-d$) shows PSD estimates for the three locations indicated by the markers in Figure~\ref{fig:p1spatial}($a$). At $P_1$, the location closest to the sphere the high-mode instability is dominant for the natural response, with a broader peak observable around the sub-harmonic of the instability. By $P_2$, while the high-mode instability remains visible for the natural response, the sub-harmonic has become dominant. Lastly, by $P_3$, the spectral power of both frequencies is reduced for the natural response, with an almost monotonic decrease in power from the sub-harmonic frequency.

Figure~\ref{fig:p1spatial}($b-d$) also show the PSD estimates for the three imposed rotation conditions of particular interest. Evidently, imposed rotation at all three forcing frequencies significantly affects the power spectrum. For the three imposed rotation conditions there are distinct, sharp
peaks in the power spectrum at the forcing frequency at all three locations in the shear layer (figure~\ref{fig:p1spatial}($b-d$)). The spatial distribution of spectral power
of the forcing frequencies is shown in figure~\ref{fig:rotfreqpow}.
The spectral power of the forcing frequencies is distributed across
approximately the same spatial region as the combination of both the
high-mode instability and its sub-harmonic observed for the natural response, with the region of peak power at
approximately $x/D=0.4$ for all three forcing frequencies. The large spatial domain over which the
imposed forcing frequencies are dominant in the wake, along with the lack of a spectral peak at the natural instability frequency, suggests that
the high-mode instability is suppressed, with the shedding of vortex structures locked to the forcing frequency, and that no
subsequent reduction of the dominant frequency to the first
sub-harmonic, as seen for the natural response, occurs. \js{So, what are we to conclude from this? Is it surprising that the 
natural shear layer frequency(?) is suppressed at these frequencies or would this occur for any imposed frequency at this amplitude?}
%
\begin{figure}
	\centering
	\includegraphics{p1spatial.png}
	\caption{($a$) Spatial distribution of spectral power for $St=3.87$ and $St=1.93$ for the natural response. ($b, c, d$) PSD estimate of cross-stream velocity, averaged over two TR-PIV data sets and the two symmetric locations indicated by the grey circles. Each PSD estimate is separated by three decades. The dashed lines indicate the two frequencies shown in ($a$). The coordinates of the markers are ($b$) $x/D=0.3$ and $y/D=\pm 0.55$, ($c$) $x/D=0.5$ and $y/D=\pm 0.575$, ($d$) $x/D=0.7$ and $y/D=\pm 0.60$.}
	\label{fig:p1spatial}
\end{figure}
%
\begin{figure}
	\centering
	\includegraphics{rotfreqpow.png}
	\caption{Spatial distribution of spectral power of the forcing
		frequency. The two dotted lines in ($a$)
		indicate the location of sampling in
		figure~\ref{fig:vh_0p35} and figure~\ref{fig:vh_0p9}. The
		dotted rectangle in ($c$) indicates the spatial domain of
		figure~\ref{fig:f21snapshotzoom_avg}.}
	\label{fig:rotfreqpow}
\end{figure}
%

In an effort to link the effect of forcing observed in the power
spectra to any changes in the wake structures, the evolution of
vortical structures near separation was tracked over time.
Figure~\ref{fig:vh_0p35} shows contours of both the out-of-plane
vorticity component, \wz~$=\omega_zD/U$, and vortex boundaries
identified using the \gtwo~criterion \citep{graftieaux2001} at
$x/D=0.35$ (location shown in figure~\ref{fig:rotfreqpow}($a$)). For
the natural response, vortex structures can be observed convecting
past $x/D=0.35$ at the high-mode instability frequency. It is worth
noting once more, that the TR-PIV provides a `cut' through the
strongly three-dimensional sphere wake and that positive and negative
regions of rotation indicated in figure~\ref{fig:vh_0p35} are unlikely
to be separate \js{ or unconnected?} structures. \citet{Yun2006} showed that for a fixed
sphere, without rotation at a similar Reynolds number
($1\times10^4$), vortex rings shed from the sphere with the central
axis of the ring initially parallel to the free-stream flow. From
figure~\ref{fig:vh_0p35}($a$) \js{the unperturbed case} the phase between regions of positive
and negative rotation convecting past $x/D=0.35$ appears intermittent.

The shedding of periodic vortex structures remains visible for all
three forcing conditions implemented at $\Ustar=15$
(figure~\ref{fig:vh_0p35}). For $\freqrat=21$ and
$\freqrat=26$, distinct structures shedding at the imposed forcing
frequency can be observed. There is no indication that coherent
structures are shedding at the natural high-mode instability
frequency, as observed for the natural response. This indicates that
the high-mode instability in the wake is suppressed by the forcing. For $\freqrat=16$ however, the wake is less periodic,
structures form at both the forcing frequency and the natural
high-mode instability frequency. This suggests that the
high-mode instability is only intermittently suppressed by the forcing
for $\freqrat=16$. For
$\freqrat=21$ and $\freqrat=26$, where structures are consistently
shed at the imposed forcing frequency, it can be seen from
figure~\ref{fig:vh_0p35} that regions of positive and negative
rotation are consistently convecting past $x/D=0.35$ alternately on either side
of the sphere. Lastly, for $\freqrat=21$ and $\freqrat=26$ in particular, it is apparent that the structures shed
are significantly larger than for the natural response.
%
\begin{figure}
	\centering
	\includegraphics{vh_0p35.png}
	\caption{Spatio-temporal distribution of ($a-d$) \gtwo~and
		($e-h$) \wz~at $x/D=0.35$ (location shown in
		figure~\ref{fig:rotfreqpow}($a$)). ($a,e$)
		$\freqrat=0$, ($b,f$) $\freqrat=16$, ($c,g$)
		$\freqrat=21$, ($d,h$) $\freqrat=26$. Single level
		contours of $\gtwo= \pm 2/\pi$ are shown. Contour limits of
		\wz~are [-0.6, 0.6]. Blue contours show clockwise and red contours show anti-clockwise \wz~and \gtwo.}
	\label{fig:vh_0p35}
\end{figure}
%

Further downstream at $x/D=0.9$, the spatio-temporal plots of \gtwo~still show periodic structures convecting downstream, albeit less distinctly (figure~\ref{fig:vh_0p9}). For the natural
response, structures around the sub-harmonic of the
high-mode instability are visible. This observation concurs with the
the spatial distribution of the power spectra shown in
figure~\ref{fig:p1spatial} and suggests that vortices are pairing. Vortex pairing in the wake of a sphere has been observed both at lower and higher Reynolds numbers previously \citep[e.g.,][]{Rodriguez2011,Bakic2006}. With imposed rotation, distinct vortex structures can be seen convecting
downstream periodically at the forcing frequency. Interestingly, for $\freqrat=16$ the forcing
frequency has now become more evident, whilst for $\freqrat=26$ the
structures have become less periodic. It appears that for
$\freqrat=16$, although the high-mode instability was only
intermittently locking to the forcing frequency at $x/D=0.35$, the
natural process by which the dominant frequency is reduced to the
sub-harmonic still occurs here and by $x/D=0.9$, more structures are
visible at a wave length close to the forcing frequency. Results for $\freqrat=10$, not shown here, indicate that no lock-on behaviour occurs for the lower forcing frequency.
 

%
\begin{figure}
	\centering
	\includegraphics{vh_0p9.png}
	\caption{Spatio-temporal distribution of ($a-d$) \gtwo~ and
		($e-h$) \wz~at $x/D=0.9$ (location shown in
		figure~\ref{fig:rotfreqpow}($a$)). ($a,e$)
		$\freqrat=0$, ($b,f$) $\freqrat=16$, ($c,g$)
		$\freqrat=21$, ($d,h$) $\freqrat=26$. Single level
		contours of $\gtwo= \pm 2/\pi$ are shown. Contour limits of
		\wz~are [-0.6, 0.6]. Blue contours show clockwise and red contours show anti-clockwise \wz~and \gtwo.}
	\label{fig:vh_0p9}
\end{figure}
%

To indicate of the phase of the cross-stream velocity
across the spatial domain a technique similar to spectral proper orthogonal decomposition (SPOD) was used. SPOD is a space-time formulation of proper
orthogonal decomposition (POD), where the analysis is conducted on
data in the frequency domain. \citet{Towne2018} provide a thorough
description of the technique. Here, the following process was conducted for each control condition. Using two data-sets, the velocity at each spatial location was first decomposed into a series of Hanning windows. The fast Fourier transform of the combined set of windowed data was then determined. Lastly, a singular value
decomposition was used on the frequency-domain data. This technique provides a
series of modes that oscillate at a single frequency. As for spatial-
only POD, the first mode provides the best first-order
reconstruction of the flow field. Each subsequent mode accounts for
progressively less variance in the data. Here, the first mode
accounted for 31\% of the variance for the natural response, and the majority of variance for the three imposed control conditions: 78\% for $\freqrat=16$,
75\% for $\freqrat=21$, and 62\% for $\freqrat=26$. An indication of the dominant phase distribution of the cross-stream velocity across the spatial domain is evident in the phase reconstructed from the Fourier coefficients of the first mode as shown in figure~\ref{fig:phase}. Since for the three imposed control conditions the majority of variance is contained in the first mode, the reconstruction should provide a good indication of the spatial phase distribution. A comparison of the spatial phase modes obtained for the flow structures observed in the raw velocity fields was made to ensure
that the phase indeed exemplifies the actual flow structures.

Figure~\ref{fig:phase} shows the phase distribution of the cross-stream velocity for the sub-harmonic
of the high-mode instability for the natural response, and for the
imposed forcing frequency for the three control conditions. In addition to
colour contours depicting phase, the
transparency across the spatial domain was set based on the spectral
power of the depicted frequency. The phase maps indicate that the vortex
structures on either side of the shear layer remain close to \ang{180} out of phase for $\freqrat=21$ and
$\freqrat=26$ up to $x/D\approx1.2$, whereafter the phase becomes less
distinct. For $\freqrat=16$, the dominant mode of the phase is closer
to \ang{90} between either side of the shear layer. For
the natural response, whilst the phase between either side of the shear layer appears to be approximately \ang{180}, there is significantly less energy associated with the first mode indicating that there is a less stationary
coherent phase pattern.
%
\begin{figure}
	\centering
	\includegraphics{Phase.png} %PIV Computer
	\caption{Phase of the first mode for: ($a$)
		$St=1.93$; ($b$) $St=1.07$;
		($c$) $St=1.40$; and ($d$)
		$St=1.74$. Transparency is set based on the spectral power of the frequency.}
	\label{fig:phase}
\end{figure}
%

\subsubsection{Phase-averaged wake and comparison to circular cylinder geometry}
For $\freqrat=21$, it was observed that the generation of coherent structures in the shear layer is locked to the forcing frequency, with a structure shedding
each rotatory oscillation cycle. In an attempt to reveal the process
by which these structures are generated and shed downstream, phased
averaged particle image velocity (PIV) over the spatial region near separation on one side of the sphere (region shown in
figure~\ref{fig:rotfreqpow}($c$)) is
analysed (figure~\ref{fig:f21snapshotzoom_avg}). For a rotary
oscillating circular cylinder, \citet{Shiels2001} examined the
vorticity generation and subsequent shedding of vortex structures. Their two-dimensional computational simulations were conducted
at $Re=1.5 \times 10^4$, where it was noted that the flow would be
three dimensional. However, the simulations appear to provide a
good representation of the critical flow physics in the boundary layer
and near wake. Clearly there are significant differences
between the two studies, especially in their not modelling  the three-dimensional effects
and the fact that, here, the high frequency oscillations are
predominantly interacting with small-scale shear-layer structures as
opposed to large-scale wake structures. However, there appear to be
similarities in the generation process of the vortex structures. \citet{Shiels2001} noted that the
generation and subsequent shedding of a vortex structure occurs at a
counter-intuitive time in the cylinder oscillation cycle. They
observed that a vortex structure is generated and separates from the
cylinder on the side of the cylinder that is moving with the
free-stream flow. This process was attributed to the transient effects
of the rotary oscillations combined with advection effects, causing
the existence of opposite-signed vorticity layers near the cylinder
wall. Likewise, here for a fixed sphere, it appears that a vortex
structure is primarily generated and separates from the sphere during
the phase of oscillation where that side of the sphere is moving with
the free-stream flow (figure~\ref{fig:f21snapshotzoom_avg}($a-c$)). 
\js{Of course this will affect vorticity generation as well. Given that
the acceleration of a surface relative to the fluid generates vorticity, as per 
Morton's paper, then there are significant differences in when vorticity of either
sign is being created over the period of the oscillation}
Unlike for the circular cylinder though, there is minimal variation in the
location of separation over an oscillation cycle and there does not
appear to be the generation of a multipole vortex structure. Rather,
during the phase of oscillation when the side of the sphere is moving
against the free-stream, it appears that flow is drawn back towards
the separation point from the re-circulation region, further
strengthening the recently generated vortex and aiding its liftoff
from the surface and convection downstream
(figure~\ref{fig:f21snapshotzoom_avg}($d-a$)). Where the high-mode instability locks to the forcing frequency, this
pattern is highly periodic and explains why the vortex structures emanating
from the shear layer shed out of phase on either side of the sphere as
the sphere oscillates. Lastly, the equatorial plane phase averaged PIV presented here
shows the maximum tangential acceleration between the sphere and
fluid. Moving away from the equatorial plane, the tangential
acceleration decreases, and likely the associated generation of
vorticity along with it. \js{As per the note above - reference to Morton's paper may be appropriate.}

Figure~\ref{fig:f21snapshotzoom_avg} also highlights that with imposed
rotation, vortices are generated by a different mechanism than for the
natural response. For the natural response, a Kelvin-Helmholtz like
instability results in the growth of disturbances, in an initially
laminar shear layer, that results in the formation and roll-up of distinct
vortex structures some distance downstream of separation. With imposed
control, the oscillatory tangential acceleration between the body and
fluid results in the generation of vortex structures near separation very close to the body.
%
\begin{figure}
	\centering
	\includegraphics{f21snapshotzoom_avg_warrow.png}
	\caption{Phase averaged vorticity and \gtwo~for $\freqrat=21$
		at $\Ustar=15$. ($a$) $t/T=0$, ($b$) $t/T=0.25$, ($c$)
		$t/T=0.5$, ($d$) $t/T=0.75$. The reduced spatial domain
		shown here is indicated in figure~\ref{fig:rotfreqpow}($c$).
		Black lines show contours of $\gtwo= 2/\pi$. Colour contours show \wz~, the limits are [-0.6, 0.6].  Blue contours show clockwise and red contours show anti-clockwise \wz. The vertical orange line on the sphere
		indicates $\theta = \ang{0}$ and the white lines show phase averaged
		$\theta$.}
	\label{fig:f21snapshotzoom_avg}
\end{figure}
%

While the imposed rotary oscillations significantly alter
the characteristics of the high-mode instability, it has been shown
that in the case of an elastically mounted sphere, the large-scale,
low-mode instability primarily produces the force that
sustains VIV. At the furthest downstream position ($x/D=1.5$), albeit still
relatively close to the sphere, in the wake centre ($y/D=0$) the
low-mode instability frequency is discernible for all conditions
examined. The spatial distribution does not change significantly with imposed
rotation. As discussed in \S~\ref{sec:Intro}, \citet{Yun2006} proposed that the large-scale
waviness of the wake (i.e., low-mode instability) is formed by a
periodic tilting of the shear-layer vortices (i.e., high-mode
instability) due to differing convection velocities around the sphere.
Here, it would seem that given the lock-on of the shear layer
vortices to the imposed rotation, that the natural tilting of the
vortices would be suppressed leading to a suppression of the large-scale wake
waviness. From analysis not shown here, with imposed rotation, the low-mode instability was indeed slightly suppressed. In the current literature, a thorough explanation of the relationship, if any, between the high- and low-mode instabilities in the wake of a fixed, stationary sphere remains elusive. \js{Do you want to contrast this with the cylinder, where there has been considerable work done on the relationship - from Bloor to Thompson \& Hourigan} Clearly, the addition of imposed rotation has not simplified the matter, with no obvious pattern emerging unlike for the high-mode instability. Nonetheless, the relationship between the two instabilities will be discussed further in \S~\ref{sec:Discussion}.
%
%\begin{figure}
%	\centering
%	\includegraphics{p1spatial_LM_ontop.png}
%	\caption{($a$) PSD estimate of cross-stream velocity, averaged over
%          two TR-PIV data sets at $P_4$ (shown in ($b$)). Dashed line shows the low-mode instability for the natural response. ($b$) The spatial
%          distribution of spectral power for $St=0.2$ for the natural response.}
%	\label{fig:p1spatial_LM}
%\end{figure}
%%

In this section, an examination of a few select control conditions (i.e., $\freqrat=16$, $\freqrat=21$, and $\freqrat=26$) for a fixed sphere
at a Reynolds number equivalent to an elastically mounted sphere in
the mode III transition regime ($\Ustar=15$) was conducted. A correlation between
VIV suppression and a suppression of the high-mode instability and associated lock-on of shear-layer structures to the
imposed rotary oscillations was shown. To confirm that the lock-on
behaviour occurs when the mounting is changed to an elastic configuration, we now examine a sphere undergoing VIV.

\subsection{Elastically mounted sphere} \label{sec:free}%
%With the high-mode instability likely largely
%unaffected by the elastic mounting and the low-mode instability locked
%to the natural frequency of the system, the optimal rotation frequency
%is much closer to the high-mode instability in the mode II and mode
%III transition regimes. As such, it is conjectured that the rotary
%oscillations are likely interacting with the high-mode, small-scale
%instability close to the sphere in a manner that, in-turn, affects the
%formation of the larger-scale wake structures that sustain vibration.

%%%%%

Less work has been conducted on the presence of the high-mode
instability for an elastically mounted sphere. \citet{vanHout2013}
used TR-PIV to investigate the near-wake of a tethered sphere over the
range $493\leqslant Re\leqslant2218$. At the highest Reynolds number
investigated, they observed a weak, broad peak in the power spectrum
close to the high-mode instability observed for a fixed sphere. Due to
the lower Reynolds number investigated (i.e., approximately an order
of magnitude lower than the results examined here for $\Ustar=15$),
they did not observe the shedding of distinct periodic structures
close to the sphere, like those observed for the fixed sphere in
section~\ref{sec:fixed} and that might be expected with elastic
mounting in the mode III transition regime.

Here we again focus on results for $\Ustar=15$, where the vibration response was suppressed, and examine imposed forcing for $\freqrat=16$, $\freqrat=21$, and $\freqrat=26$. 
Figure~\ref{fig:freef0f21snapshot} shows instantaneous snapshots of velocity vectors, along with out-of-plane vorticity and \gtwo~contours over a single vibration cycle for both the natural response and $\freqrat=21$.  
This figure highlights the difficulty in comparing the wake between control conditions for an elastically mounted body.  
Specifically, whilst similarities in small-scale wake structures are evident, the difference in vibration amplitude significantly alters the overarching wake structure in relation to sphere position.  
Importantly though for the comparison between fixed and elastic mounting, with imposed rotation small-scale structures do appear to shed periodically from the sphere, with similar size and frequency to the equivalent fixed configuration.  
%
\begin{figure}
	\centering
	\includegraphics{freef0f21snapshot.png}
	\caption{Instantaneous snapshots of velocity vectors (every
		fifth vector shown for clarity), vorticity colour contours,
		and \gtwo~line contours for ($a$) $\freqrat=0$, and ($b$)
		$\freqrat=21$. Blue (clockwise) and red
		(anti-clockwise) lines show contours of $\gtwo= \pm 2/\pi$.
		Colour contour limits of \wz~are [-0.7, 0.7]. For ($b$), the
		vertical orange line on the sphere indicates $\theta = \ang{0}$ and the
		white lines show phase averaged $\theta$.}
	\label{fig:freef0f21snapshot}
\end{figure}
%

Figure~\ref{fig:vh_free_0p55} shows the evolution of \gtwo~and \wz~past $x/D=0.55$. 
As for the fixed sphere, the high-mode instability appears suppressed for both $\freqrat=21$ and $\freqrat=26$, particularly on the trailing side of the sphere (i.e., between approximately $0-2$ s for $\freqrat=21$). 
Once more, the high-mode instability appears to be intermittently suppressed for $\freqrat=16$. 
\js{I found this a bit difficult to interpret. I can see that the red contours (-ve) are smaller in amplitude than the blue (+ve). Also, that the low frequency has less effect but it is hard to discern the intermittency for the $f_r = 16$ case. I there a way of pointing the reader in the right direction?}
Unlike the natural response (figure~\ref{fig:vh_free_0p55}($a,e$)), with imposed rotatory oscillations, shear-layer structures convect past $x/D=0.55$ relatively consistently over the entire vibration cycle. 
While clearly interesting, it is difficult to ascertain whether distinct structures are more observable where suppression occurs because of the inherent effect of the rotary oscillations on the flow, or merely because the vibration is more suppressed and as such, the wake more resembles that of the fixed sphere.
%
\begin{figure}
	\centering
	\includegraphics{vh_free_0p55.png}
	\caption{Spatio-temporal distribution of ($a-d$) \gtwo~ and
		($e-h$) \wz~at $x/D=0.55$. ($a,e$) $\freqrat=0$, ($b,f$)
		$\freqrat=16$, ($c,g$) $\freqrat=21$, ($d,h$)
		$\freqrat=26$. Single level contours of $\gtwo= \pm 2/\pi$
		are shown. Contour limits of \wz~are [-0.6, 0.6]. Blue contours show clockwise and red contours show anti-clockwise \wz~and \gtwo.}
	\label{fig:vh_free_0p55}
\end{figure}
%%
%\begin{figure}
%	\centering%
%	\includegraphics{vh_free_1p2.png}
%	\caption{Spatio-temporal distribution of ($a-d$) \gtwo~ and ($e-h$) \wz~at $x/D=1.2$. ($a,e$) $\freqrat=0$, ($b,f$) $\freqrat=16$, ($c,g$) $\freqrat=21$, ($d,h$) $\freqrat=26$. Single level contours of $\gtwo= \pm 2/\pi$ are shown. Contour limits of \wz~are [-0.6, 0.6]. Blue contours show clockwise and red contours show anti-clockwise \wz~and \gtwo.}
%	\label{fig:vh_free_1p2}
%\end{figure}
%%

To quantify the frequency and periodicity of the shear-layer shedding in the wake of an elastically mounted sphere, a technique similar to that used by \citet{vanHout2013} was employed to track the shear-layer position. 
The $y$-direction position of maximum out-of-plane vorticity at $x/D=0.6$ was determined for each time-step where $\wz>0.4$.
Where $\wz<0.4$, the shear-layer position was interpolated. 
A 15 Hz low-pass filter was then applied to the tracked shear-layer position. 
The cross-stream velocity at the tracked shear-layer position at each time-step was then used to obtain an estimate of the power spectrum in the moving shear layer. 
Using \gtwo~to track the shear layer position gave similar results. 
%%
%\begin{figure}
%	\centering
%	\includegraphics{psdmethod.png}
%	\caption{Example of shear layer tracking for $\freqrat=21$.
%          Black line indicates tracked shear layer position used for
%          PSD estimate.}
%	\label{fig:psdmethod}
%\end{figure}
%%

Figure~\ref{fig:psdtrackontop} shows the resultant power spectra at $x/D=0.6$. 
For the natural response, a peak in the power spectrum is observed at the same frequency as the high-mode instability identified for the fixed sphere. 
A second less distinct peak, slightly higher than the sub-harmonic of the high-mode instability can also be observed. 
For $\freqrat=16$ and $\freqrat=21$, a strong peak at the forcing frequency is visible. 
For $\freqrat=26$ on the other hand, there is no strong peak evident in the spectrum at the forcing frequency. 
This observation is in line with the results from figure~\ref{fig:p1spatial} where there was a larger reduction in the spectral power of the forcing frequency for $\freqrat=26$ between $x/D=0.3$ and $x/D=0.7$ than for the other control conditions. 
From figure~\ref{fig:psdtrackontop}($b$), it is evident that forcing at $\freqrat=21$ results in the most distinct spectral peak, suggesting that periodic shedding from the shear layer is strongest for this control condition.
%
\begin{figure}
	\centering
	\includegraphics{psdtrackontopsbs.png}
	\caption{PSD estimate of cross-stream velocity in tracked shear layer
		at $x/D=0.6$. ($a$) Each PSD estimate is separated by three decades.
		($b$) No separation between PSD estimates. Blue dashed
		lines shows the high-mode instability and its
		sub-harmonic for the natural response of the fixed sphere.}
	\label{fig:psdtrackontop}
\end{figure}
%

TR-PIV acquired at $\Ustar=12$ and $\Ustar=18$ indicated that lock-on of vortex shedding in the shear layer to the forcing frequency also occurred at these conditions.

It was shown in section~\ref{sec:fixed} that while the high-mode instability is dominant immediately downstream of separation, the sub-harmonic of the instability quickly dominates downstream. 
While it was difficult to ascertain whether the same behaviour happens for an elastically mounted sphere, evidence of distinct shear-layer structures persisting to at least $x/D=1.2$ were observed when the lock-on phenomenon occurred. 
With elastic mounting, it is difficult to determine the effect of forcing on the low-mode instability due to the coupled sphere movement and lock-in behaviour. 
Some further discussion on this is presented in \S~\ref{sec:Discussion}. 

In summary, as for the fixed sphere, a suppression of the high-mode instability and a lock-on of vortex shedding to the imposed forcing was observed for control conditions where VIV is suppressed. 
In particular, lock-on, and suppression of the natural high-mode instability, was observed for forcing frequencies slightly lower than the sub-harmonic of the high-mode instability. 
Forcing at frequencies close to and lower than the high-mode instability sub-harmonic appeared to result in the generation of vortices which persisted the furthest downstream resulting in the greatest influence on the larger-scale lift-generating streamwise vortex structures.  

\subsection{Discussion on the effects of imposed forcing on wake instabilities} \label{sec:Discussion}
With the combination of a highly three-dimensional wake, transverse body vibration, and imposed rotary oscillations, it is difficult to isolate and analyse particular aspects of the flow for the elastically mounted, rotary oscillating sphere. 
To reduce this complexity, an effort was made to study the effects of forcing on a fixed sphere. 
This enabled various characteristics of the spatial dynamics and flow instabilities in the equatorial plane to be determined. 
To understand the effects of the strong three-dimensional aspects of the flow and the characteristics of the instabilities observed in relation to existing literature, we found it valuable to compare the results obtained here to those found for a two-dimensional circular cylinder.

As discussed in \S~\ref{sec:Intro}, low- and high-mode instabilities exist in the sphere wake. 
\citet{Kim1988} found a scaling between the high-mode ($f_{HM}$) and low-mode ($f_{LM}$) instabilities of approximately $f_{HM}/f_{LM} \propto Re^{0.75}$ over the range $10^3 < Re < 10^4$ and approximately $f_{HM}/f_{LM} \propto Re^{0.66}$ over the range $10^4 < Re < 10^5$. 
Comparing these results to the relationship identified between the Bloor-Gerrard shear-layer instability ($f_{SL}$) and K\'arm\'an vortex shedding ($f_K$) in the wake of a circular cylinder reveals interesting similarities. 
\citet{Thompson2005} proposed that the two instabilities observed in the circular cylinder wake, scaled as; $f_{SL}/f_{K} \propto Re^{0.57}$ over the range $1.5\times10^3 < Re < 5\times10^3$, and $f_{SL}/f_{K} \propto Re^{0.52}$ over the range $1\times10^4 < Re < 5\times10^4$. 
The similarity between the Reynolds number ranges for the two geometries is interesting as is that both the magnitude of scaling and percentage reduction in scaling from the low to high Reynolds number ranges are also similar (i.e., 9\% reduction for the circular cylinder and 12\% reduction for the sphere). 
\citet{Thompson2005} presented the distinct reduction of the wake formation length between $Re=4\times 10^3$ and $Re=1\times 10^4$ and the associated enhanced interaction between the K\'arm\'an vortices and the shear layers at high Reynolds numbers as a reason for the change in scaling between the instabilities. 
Similarly, for the sphere a striking change in formation length can be seen between $Re=3.7\times 10^3$ and $Re=1\times 10^4$ in the large eddy simulations of \citet{Yun2006} (this change in formation length has also been observed over similar Reynolds number ranges in other studies such as \citet{Jang2007}). 
They found that the re-circulation bubble approximately halves in size between $Re=3.7\times 10^3$ and $Re=1\times 10^4$. 
Therefore, it seems reasonable to postulate that, as suggested by \citet{Thompson2005} for the circular cylinder, above $Re\approx 1\times10^4$, there is enhanced interaction between the low- and high-mode instabilities of the sphere. 
Here, vibration suppression was observed for $\Ustar\geq10$ where $Re\geq1.3\times10^4$.

\citet{Tokumaru1991} examined the effect of rotary oscillations on the wake of a fixed circular cylinder over a broad forcing frequency range at $Re=1.5\times10^4$. 
They identified four wake modes. 
The flow visualisations they presented are for rotation amplitudes at least an order of magnitude larger than implemented here, and the majority are closer to two orders of magnitude larger. 
So it is not expected that the rotations imposed here will have the same large-scale effects seen by \citet{Tokumaru1991}. 
Yet, similarities are again evident between the two geometries. 
The forcing frequencies for which vibration suppression was found in the mode III transition regime in this study, fall into the mode III wake regime of \citet{Tokumaru1991}. 
In this regime for the circular cylinder, the imposed oscillations suppress both the shear-layer and K\'arm\'an instabilities, with vortex structures only being shed at the forcing frequencye. 
The wake is also significantly narrowed. 
For the sphere, likely due to the lower rotation amplitudes imposed rather than a global modification of the wake, only localised suppression of the high-mode instability in the shear layer was observed. 
Yet a narrowing of the wake, most significant for the most effective forcing frequency was observed. 
In addition to the similarities between wake width, it appears that like for the circular cylinder, as a result of the imposed forcing the onset of the low-mode instability is delayed. 

\citet{Tokumaru1991} also conducted flow visualisation at a lower Reynolds number of $Re=3.3\times10^3$. 
They found very similar results, but noted that the effects at higher forcing frequencies were more pronounced. 
\citet{Shiels2001} also found the effects of imposed forcing strongly Reynolds number dependent over the range investigated here. 
Evidently, the variation in interaction between the K\'arm\'an and shear-layer instabilities described by \citet{Thompson2005}, over the Reynolds number range used here, and by \citet{Tokumaru1991} and \citet{Shiels2001}, affects the ability of the imposed rotary oscillations to influence the wake at lower Reynolds numbers. 

In addition to the effect of forcing on the natural instabilities of the sphere wake, the mechanisms leading to sustained vibration are also likely play an important role in vibration suppression. 
\citet{Govardhan2005} identified mode III as a `movement-induced excitation' mode, whereby initial perturbations of the sphere lead to the generation of self-sustaining vortex forces. 
Any disruption to the timing of vortex generation or small disturbances to sphere displacement are more likely to lead to significant vibration suppression in this regime. 
The mode III transition regime, where vibration was suppressed in this study, is the beginning of the movement-induced excitation regime for the sphere mounted with 1DOF transverse to the oncoming flow.

Drawing together these comparisons to past work on the instabilities and forcing of a circular cylinder along with the results analysed in this study, some likely causes of vibration suppression in the mode III transition regime can be identified. 
It is clear from both the fixed and elastically mounted results that where suppression of vibration was observed the imposed rotation generates strong persistent vortex structures that shed at the forcing frequency and predominately convect downstream in the shear layer. 
Where this lock-on occurs, the excitation is strong enough to bypass the formation and pairing of the initial shear-layer vortices (high-mode instability) with only vortex structures convecting at the forcing frequency being observed. 
The vortex structures generated are larger and persist further downstream than for the natural response, with the forcing frequency remaining dominant over a large spatial domain. 
With fixed mounting at least, the wake narrows with imposed forcing. 
Furthermore, there is significant variation in the phase between the sphere displacement and vortex force acting on the sphere for the most effective forcing frequency. 
It appears that with imposed oscillations, the vortex structures persist far enough downstream to interfere with the large-scale vortex-shedding process (low-mode instability). 
As mode III vibration is due to a movement-induced instability and relies on the the natural interaction between the sphere and large-scale vortex generation, this disturbance of large-scale shedding appears to disrupt the movement-induced vibration mechanism, leading to significant vibration suppression.

The TR-PIV results along with the comparison to the circular cylinder suggest that, in addition to the imposed oscillation frequency and amplitude, both Reynolds number and vibration mode are important in determining the vibration response. 
While it is likely to always be difficult to suppress vibration in the mode I and mode II regimes, in this study the difficulty is compounded by the low Reynolds number ($Re<1\times10^4$) and associated reduced interaction between the high- and low-mode instabilities, which reduces the control effectiveness in the mode I and the beginning of mode II regimes. 

\section{Conclusions}
\label{sec:concl}
\js{Edited up to here - Conclusions not done, as per discussion with Tom}

High-frequency low-amplitude rotary oscillations have been imposed on
an elastically mounted sphere. It was found that, with careful
frequency selection, it was possible to substantially suppress VIV in
the mode II and mode III transition regimes with rotation ratios as
low as $\velrat=0.1$. At $\Ustar=18$, the highest reduced velocity
investigated, a reduction in the amplitude response of up to 84\% for
a rotation ratio of only $\velrat=0.1$ was observed. Across the
parameter space investigated, no appreciable deviation in the
vibration frequency from the natural response occurred. In regard to
the amplitude of oscillations, it was found that no further
significant suppression of the vibration occurred for rotation ratios
above $\velrat=0.3$, at least up to the maximum rotation ratio tested
($\velrat=1$). Above $\velrat=0.3$, vibration suppression was not
sensitive to rotation frequency.

To examine the effect of the imposed rotation at a low rotation ratio
($\velrat=0.1$) on the wake dynamics, three forcing conditions, in
addition to the natural response, were studied in detail for
$\Ustar=15$ for an elastically mounted sphere as well as a fixed
sphere at an equivalent Reynolds number. For the natural response, it
was found that while the high-mode instability was dominant
immediately down-stream of separation, the sub-harmonic of the
instability quickly becomes dominant. The
high-mode instability could be suppressed by imposed rotation, with
shedding of distinct, periodic vortex structures at the forcing
frequency evident. Moving downstream, no reduction in dominant
frequency was observed with imposed rotation, unlike the natural
response. While a direct connection between the altered high-mode
instability and the large-scale motions of the wake was not formally
established, it appears that with imposed oscillations, the vortex
structures generated persist far enough downstream to interfere with
the large-scale vortex-shedding process (low-mode instability), which
in-turn disrupts the movement-induced vibration mechanism leading to
significant vibration suppression. Where the vibration was suppressed,
significant variation in the instantaneous phase between the sphere
displacement and vortex force acting on the sphere was also observed.

This investigation shows promise for suppression of VIV of
three-dimensional geometries through sinusoidal rotation at
frequencies many times higher than the natural frequency of the system
and at lower amplitudes than implemented in previous studies. It also
demonstrates that VIV may be suppressed through direct interaction
with the high-mode instability as opposed to the low-mode instability
as performed in previous studies. Due to limitations of the
servo-motor used, it was not possible to impose rotary oscillations at
the expected shear-layer instability frequency when testing at high
reduced velocities. It may prove interesting to investigate the effect
of rotary oscillations at frequencies at and above the initial
shear-layer instability, at the reduced velocities where suppression
of the vibration was observed.

\section*{Acknowledgements}
TM acknowledges the financial support of an Australian Government
Research Training Program Scholarship. We would also like to
acknowledge partial support and maintenance of the experimental
facility through ARC discovery grants DP150102879, DP170100275, and
DP190103388.

\bibliography{jfs-lib}

\end{document}
